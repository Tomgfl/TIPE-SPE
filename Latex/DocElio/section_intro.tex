%%%%%%%%%%%%%%%%%%%%%%%%%%%%%%%%%%%%%%%%%%%%%%%%%
\section[Première section]{Introduction}

%++++++++++++++++++++++++++++++++++++++++++++++++
\subsection{Principe de Ray Marching}


%------------------------------------------------
\begin{frame}
\frametitle{Ce qui apparaît dans l'en-tête}

Dans la \important{première ligne}:
\begin{itemize}
\flch la version courte du titre, précisée en option  de  \texttt{$\backslash$title}\\
\remarque{( en option = entre crochets, avant les accolades)}
\flch la version courte du nom, voire des initiales, redéfinir la commande
\texttt{$\backslash$newcommand\{$\backslash$initiales\}\{Petit Nom \}} 
\flch la version courte de la date, précisée en option de \texttt{$\backslash$date}\\
\end{itemize}
\bigskip
Dans la \important{deuxième ligne}:
\begin{itemize}
\flch le numéro et le titre de la section, sauf si le numéro est nul\\
\remarque{s'il est précisé en option de} \texttt{$\backslash$section}
\remarque{le titre court est utilisé}
\flch le numéro et le titre de la sous-section, sauf si le numéro est nul\\
\remarque{s'il est précisé en option de} \texttt{$\backslash$subsection}
\remarque{le titre court est utilisé}
\end{itemize}

\end{frame}


%++++++++++++++++++++++++++++++++++++++++++++++++
\subsection{La modélisation dans le jeu vidéo}
%------------------------------------------------
\begin{frame}
\frametitle{titre de la slide sans lettre descendant sous la baseline}
pour régler ce problème, utiliser la commande \texttt{$\backslash$esp} à la fin du titre, cf slide suivante
\end{frame}


%------------------------------------------------
\begin{frame}
\frametitle{titre de la slide sans lettre descendant sous la baseline\esp}
ici c'est mieux non?
\end{frame}


%------------------------------------------------
\begin{frame}[fragile]
\frametitle{titre de la slide qui marche tout seul grâce au q et au g}
\end{frame}
