\section{Troisième section : le bas des slides}
%++++++++++++++++++++++++++++++++++++++++++++++++
\subsection{Pour les références}
%------------------------------------------------
\begin{frame}[fragile]
\frametitle{Petit bandeau de citation d'une référence\esp}

Se fait à la main en utilisant la commande \texttt{$\backslash$bandeauREF}.\\
En plus il faut ajuster à la main un  \texttt{$\backslash$vspace} pour le forcer à être bien en bas...\\
\vspace*{5.2cm}
\bandeauREF{\hspace*{0.2cm}\textbf{Auteur, 2020,} 
Journal Scientifique en Open Source,
titre de l'article\\
\hspace*{0.2cm}\textbf{Autrice, 2020,} 
Journal Scientifique en Open Source,
titre de l'article\\
\\} 
\end{frame}







%++++++++++++++++++++++++++++++++++++++++++++++++
\subsection{Numérotation}
%------------------------------------------------
\begin{frame}[fragile]
\frametitle{La numérotation des diapo\esp}
Compte les diapo sans multiplicité : si le contenu d'un diapo arrive petit à petit parce qu'on a utilisé des %\mintinline{latex}{pause} 
\\[1cm]
Ne compte pas la page de titre, ni les pages de plan. \\[1cm]
Compte le nombre total de slides\\[1cm]
Si vous avez n slides bonus, vous pouvez fausser (mais rectifier) 
le nombre total de slides avec  \texttt{$\backslash$addtocounter\{framenumber\}\{-n\} } 
\end{frame}


%------------------------------------------------
\begin{frame}[fragile]
\frametitle{La dernière vraie slide\esp}
\label{derniere_slide_effective}
Elle est donc numérotée \theframenumber / \theframenumber
\end{frame}


%------------------------------------------------
\begin{frame}[fragile]
\frametitle{Slide bonus qui ne compte pas}
Elle est donc numérotée \theframenumber / \ref{derniere_slide_effective}
\end{frame}


%------------------------------------------------
\begin{frame}[fragile]
\frametitle{Encore une slide bonus qui ne compte pas}
Elle est donc numérotée \theframenumber / \ref{derniere_slide_effective}
\end{frame}
\addtocounter{framenumber}{-2} 


